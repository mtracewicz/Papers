\documentclass{article}
\usepackage{listings}
\usepackage{color}
\usepackage{amsmath}
\usepackage[polish]{babel}
\usepackage[T1]{fontenc}
\usepackage[utf8]{inputenc}

\definecolor{dkgreen}{rgb}{0,0.6,0}
\definecolor{gray}{rgb}{0.5,0.5,0.5}
\definecolor{mauve}{rgb}{0.58,0,0.82}

\lstset{frame=tb,
  language=Python,
  aboveskip=3mm,
  belowskip=3mm,
  showstringspaces=false,
  columns=flexible,
  basicstyle={\small\ttfamily},
  numbers=none,
  numberstyle=\tiny\color{gray},
  keywordstyle=\color{blue},
  commentstyle=\color{dkgreen},
  stringstyle=\color{mauve},
  breaklines=true,
  breakatwhitespace=true,
  tabsize=4
} 

\title{Segmentacja komórek na zdjęciach mikroskopowych skóry z użyciem głębokich sieci neuronowych.}
\date{2020-12-25}
\author{Michał Tracewicz}
\begin{document}
\pagenumbering{gobble}
\maketitle
\newpage
\tableofcontents
\newpage
\section{Przedstawienie problemu}
W codziennej pracy wielu lekarzy poświęca czas na manualne liczenie zafarbowanych komórek na zdjęciach.
Jest to czasochłonne i monotonne zadanie, nie wymagające sześcioletnich studiów.
Zadanie takie jak to idealnie nadaje się do automatyzacji.
W ostatnich latach bardzo mocno rozwija się użycie uczenia maszynowego do rozwiązywania tej klasy problemów.
W ramach mojej pracy powstał zestaw narzędzi pozwalający na łatwe testowanie różnych architektur, funkcji strat w tym problemie.
Dodatkowo przeprowadziłem testy dla architektury u-net oraz pewnych wariacji na jej temat.
Całość dopełnia moduł pozwalający na obróbkę zdjęć wejściowych oraz wyjściowych.
\section{Sztuczne sieci neuronowe}
Sztuczne sieci neuronowe są próbą stworzenia programów, których działanie jest inspirowane naszym rozumieniem mózgu.
Chociaż zagadnienie to podejmowano już od bardzo długiego czasu, to dopiero od niedawna posiadamy wystarczającą moc obliczeniową aby realizować tworzone od wielu dziesięcioleci algorytmy.

\subsection{Ogólne zasady działania}
Głębokie sieci neuronowe składają się z warstwy wejściowej, wyjściowej oraz pewnej liczby warstw ukrytych.
Liczba oraz typ tych ostatnich w dużej mierze determinuje działanie sieci.
Omówienie zaczniemy od tego w jaki sposób sieć przeprowadza obliczenia zwracające odpowiedź, a następnie zajmiemy się procesem uczenia.
Każda z warstw zawiera pewną ilość neuronów.
Neurony kolejnych warstw są ze sobą połączone (zajmę się tu jedynie warstwami, które posiadają połączenia neuronów każdy z każdym, czyli tak zwanymi warstwami typu "dense").
Połączenia te posiadają wagi.
Aby obliczyć wynik zwracany przez sieć rozpoczynamy obliczenia od warstwy wejściowej i obliczmy wartości neuronów w kolejnych warstwach.
Wartość dla neuronu w warstwie k-tej to wartość funkcji aktywacji na sumie wartości wszystkich neuronów z warstwy poprzedniej przemnożonych przez odpowiednie im wagi połączeń.
\begin{align*}
  x_{k,j} = \sigma(\sum\limits_{i=0}^{n-1}{w_i*x_{(k-1),i}})
\end{align*}
Gdzie (k>=1):
\begin{itemize}
  \item $\sigma$ - funkcja aktywacji
  \item $x_{k,j}$ - wartość j-tego neuronu w k-tej warstwie
  \item n - liczba neuronów w warstwie k-1
  \item $w_i$ - i-ta waga wchodząca do neuronu
\end{itemize}
\subsection{Porównanie do klasycznych metod programowania}
Przy standardowych paradygmatach programowania
\section{Sieci splotowe}
\subsection{Operacja splotu}
\subsection{Sieci w pełni splotowe}
\section{Problem segmentacja}
Zaznaczenie konturów komórek na zdjęciach mikroskopowych jest problemem segmentacji instancji.
W celu zrozumienia tego pojęcia przeanalizujemy składające się na niego mniejsze problemy z kategorii widzenia komputerowego.
\subsection{Klasyfikacja}
Jest to najprostsze zagadnienie w dziedzinie widzenia komputerowego.
Polega ono na tym, że dla otrzymanego na wejście obrazu sieć ma zwrócić klasę do której przedstawiony na zdjęciu obiekt należy.
Za dobry przykład może nam posłużyć "Problem MNiST", polegający na rozpoznawaniu odręcznie napisanych cyfr.
Jako wejście sieć otrzymuje czarno-biały obrazek rozmiaru 16x16 pikseli na, którym widnieje odręcznie napisana cyfra.
Natomiast na wyjściu sieć zwraca nam predykcję jaka cyfra widnieje na zdjęciu.
Ograniczeniem w tym typie problemu jest fakt, że na danym zdjęciu może znajdować się jedynie pojedyńczy obiekt.
\subsection{Klasyfikacja z lokalizacją}
Klasyfikacja z lokalizacją to dodanie do poprzedniego problemu zagadnienia odnalezienia klasyfikowanego obrazu na zdjęciu.
Przykładowo tworząc sieć rozpoznającą gatunki zwierząt domowych, otrzymując zdjęcie kota siedzącego na kanapie otrzymamy nie tylko informację, 
że jest to kot ale również informację o tym gdzie na tej kanapie się znajduje.
Najczęściej zwracane są w takim wypadku oprócz klasy współrzędne pikseli w których zaczyn i kończy się obiekt.
Następnie możemy na tej podstawie narysowanie naokoło obiektu ramkę z podpisem zawierającym nazwę klasy.
\subsection{Wykrywanie obrazów}
Kolejnym krokiem jest wykrywanie obrazów. Jest to klasyfikacja z lokalizacją dla wielu elementów jednocześnie.
Kontynuując poprzedni przykład z rozpoznawaniem zwierząt domowych.
W wypadku zdjęcia, które zawiera zarówno psa i kota dostaniemy zestaw współrzędnych oraz odpowiadających im klas.
\subsection{Segmentacja}
Segmentacja to przypisanie każdemu pikselowi na obrazie wejściowym klasy.
Za przykład może nam posłużyć zdjęcie drogi w mieście.
Celem naszej sieci neuronowej będzie podział obrazu na klasy: droga, człowiek, budynek, drzewo, niebo etc.
Na wyjściu możemy na przykład otrzymać ten sam obrazek z tym, że zależnie od klasy danego piksela będzie miał on różny kolor (czerwony - droga, zielony - człowiek etc.)
\subsection{Segmentacja instancji}
Segmentacja instancji to dodanie do segmentacji ograniczenia, że każda jednostka ma być widocznie oddzielona.
Dla przykładu gdy mamy zdjęcie czterech przytulających się osób, w wyniku segmentacji wszystkie cztery zostaną zakolorowane na ten sam kolor oraz podpisane raz jako człowiek.
Natomiast gdy przeprowadzimy segmentację instancji to w wyniku otrzymamy każdą osobę pokolorowaną na własny kolor, oraz etykiety "człowiek 1.", "człowiek 2." etc.
\section{Architektura U-Net}
\subsection{Historia i zastosowania}
\subsection{Obrazy wejściowe}
\subsection{Obrazy wyjściowe}
\section{Przetestowane podejścia}
\subsection{Bazowa sieć u-net}
\subsection{Preprocessing obrazów}
\subsubsection{Cięcie i sklejanie obrazów}
\subsubsection{Wybór jednego kanału - czerwony/alfa}
\subsubsection{Rozmycie - gausian blur}
\subsubsection{Przekształcenie do obrazów trzy kanałowych}
\subsection{Funkcja strat - Współczynnik Sørensena}
\subsection{Własna funkcja strat}
\subsection{Zapis wyniku w postaci distance map}
\section{Powstałe narzędzia}
\subsection{Skrypty pozwalające na preprocessing obrazów uczących}
\subsection{Skrypty do uczenie sieci, poprzez wstrzyknięcie architektury sieci/funkcji strat}
\subsection{Skrypty pozwalające na testowanie zapisanych modeli}
\subsection{Skrypty pozwalające na predykcję dla obrazu/folderu obrazów}
\subsection{Testy jednostkowe}
\subsection{Automatyczne renderowaie i publikacja wyników}
\section{Podsumowanie i możliwe następne kroki}
\section{Bibliografia}
\end{document}