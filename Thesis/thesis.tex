\documentclass{article}
\usepackage{listings}
\usepackage{color}
\usepackage[polish]{babel}
\usepackage[T1]{fontenc}
\usepackage[utf8]{inputenc}

\definecolor{dkgreen}{rgb}{0,0.6,0}
\definecolor{gray}{rgb}{0.5,0.5,0.5}
\definecolor{mauve}{rgb}{0.58,0,0.82}

\lstset{frame=tb,
  language=Python,
  aboveskip=3mm,
  belowskip=3mm,
  showstringspaces=false,
  columns=flexible,
  basicstyle={\small\ttfamily},
  numbers=none,
  numberstyle=\tiny\color{gray},
  keywordstyle=\color{blue},
  commentstyle=\color{dkgreen},
  stringstyle=\color{mauve},
  breaklines=true,
  breakatwhitespace=true,
  tabsize=4
} 

\title{Segmentacja komórek na zdjęciach mikroskopowych skóry z użyciem głębokich sieci neuronowych.}
\date{2020-12-25}
\author{Michał Tracewicz}
\begin{document}
\pagenumbering{gobble}
\maketitle
\newpage
\tableofcontents
\newpage
\section{Przedstawienie problemu}
\section{Sztuczne sieci neuronowe}
\subsection{Ogólne zasady działania}
\subsection{Porównanie do klasycznych metod programowania}
\section{Sieci splotowe}
\subsection{Operacja splotu}
\subsection{Sieci w pełni splotowe}
\section{Problem segmentacja}
\subsection{Klasyfikacja}
\subsection{Klasyfikacja z lokalizacja}
\subsection{Wykrywanie obrazów}
\subsection{Segmentacja}
\subsection{Segmentacja instancji}
\section{Architektura U-Net}
\subsection{Historia i zastosowania}
\subsection{Obrazy wejściowe}
\subsection{Obrazy wyjściowe}
\section{Przetestowane podejścia}
\subsection{Bazowa sieć u-net}
\subsection{Preprocessing obrazów}
\subsubsection{Cięcie i sklejanie obrazów}
\subsubsection{Wybór jednego kanału - czerwony/alfa}
\subsubsection{Rozmycie - gausian blur}
\subsubsection{Przekształcenie do obrazów trzy kanałowych}
\subsection{Funkcja strat - Współczynnik Sørensena}
\subsection{Własna funkcja strat}
\subsection{Zapis wyniku w postaci distance map}
\section{Powstałe narzędzia}
\subsection{Skrypty pozwalające na preprocessing obrazów uczących}
\subsection{Skrypty do uczenie sieci, poprzez wstrzyknięcie architektury sieci/funkcji strat}
\subsection{Skrypty pozwalające na testowanie zapisanych modeli}
\subsection{Skrypty pozwalające na predykcję dla obrazu/folderu obrazów}
\subsection{Testy jednostkowe}
\subsection{Automatyczne renderowaie i publikacja wyników}
\section{Podsumowanie i możliwe następne kroki}
\section{Bibliografia}
\end{document}